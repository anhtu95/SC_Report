\documentclass[a4paper,12pt]{report}
\usepackage[utf8]{vietnam}
\usepackage{graphicx}
\usepackage{fancybox}
\usepackage{longtable}
\usepackage{listings}
\usepackage{relsize}
\usepackage{cases} 
\usepackage[left=3cm, right=2.00cm, top=2.00cm, bottom=2.00cm]{geometry}
\lstset{
   %keywords={break,case,catch,continue,else,elseif,end,for,function,
   %   global,if,otherwise,persistent,return,switch,try,while},
   basicstyle=\ttfamily \fontsize{12}{15}\selectfont,   
	% numbers=left,
   frame=lrtb,
tabsize=2
}
\setlength{\parskip}{0.6em}
\usepackage{hyperref}
\usepackage{float}
\hypersetup{
    colorlinks,
    citecolor=black,
    filecolor=black,
    linkcolor=black,
    urlcolor=black
}
\usepackage[nottoc]{tocbibind}
\usepackage[english]{babel}
\usepackage{indentfirst}
\addto\captionsenglish{%
 \renewcommand\chaptername{Phần}
 \renewcommand{\contentsname}{Mục lục} 
 \renewcommand{\listtablename}{Danh sách bảng}
 \renewcommand{\listfigurename}{Danh sách hình vẽ}
 \renewcommand{\tablename}{Bảng}
 \renewcommand{\figurename}{Hình}
 \renewcommand{\bibname}{Tài liệu tham khảo}
}
\begin{document}
\thispagestyle{empty}
\thisfancypage{
\setlength{\fboxrule}{1pt}
\doublebox}{}
\begin{center}
{\fontsize{16}{19}\fontfamily{cmr}\selectfont TRƯỜNG ĐẠI HỌC BÁCH KHOA HÀ NỘI\\
VIỆN CÔNG NGHỆ THÔNG TIN VÀ TRUYỀN THÔNG}\\
\textbf{------------*******---------------}\\[1cm]
\includegraphics[scale=0.13]{hust.jpg}\\[1.3cm]

{\fontsize{32}{43}\fontfamily{cmr}\selectfont BÁO CÁO}\\[0.1cm]
{\fontsize{38}{45}\fontfamily{cmr}\fontseries{b}\selectfont MÔN HỌC}\\[0.2cm]
{\fontsize{19}{20}\fontfamily{phv}\selectfont Tính toán khoa học }\\[0.2cm]
{\fontsize{13}{20}\fontfamily{cmr}\selectfont Đề tài: Mạng Google Inception trong bài toán phân loại}\\[2.5cm]
\end{center}
\hspace{1cm}\fontsize{14}{16}\fontfamily{cmr}\selectfont \textbf{Nhóm sinh viên thực hiện:}

\begin{longtable}{l c c}

Họ và tên & MSSV  & Lớp\\
Nguyễn Tuấn Đạt & 20130856 & CNTT2.02-K58 \\
Đặng Quang Trung & 20134145 & CNTT2.02-K58 \\
Phan Anh Tú & 20134501 & CNTT2.01-K58 \\
\end{longtable}

\hspace{0.6cm}\fontsize{14}{16}\fontfamily{cmr}\selectfont \textbf{Giảng viên môn học: }TS. Đinh Viết Sang \\[1.5cm]
\begin{center}
\fontsize{16}{19}\fontfamily{cmr}\selectfont Hà Nội 1--2017

\end{center}
\newpage

\pdfbookmark{\contentsname}{toc}
\tableofcontents
\listoffigures


\phantomsection
\addcontentsline{toc}{chapter}{Lời cảm ơn}
\chapter*{Lời cảm ơn}
Chúng em xin chân thành cảm ơn Thầy giáo, TS. Đinh Viết Sang đã tận tình giảng dạy, hướng dẫn chúng em thực hiện đề tài này. Trong quá trình thực hiện, đề tài của chúng em không tránh khỏi những hạn chế, thiếu sót, chúng em rất mong nhận được những ý kiến đánh giá, nhận xét của Thầy để đề tài này có thể được hoàn thiện hơn.


\chapter{Đặt vấn đề}
Trong bối cảnh nghành công nghiệp công nghệ thông tin đang phát triển rất mạnh mẽ ngày này, có lẽ Deep Learning là một trong những từ khóa được quan tâm nhất. Deep Learning đang được ứng dụng ngày càng rộng rãi trong các công việc của cuộc sống hàng ngày như: ô tô tự lái, nhận diện giọng nói, nhận diện khuôn mặt ....
\par Là một sinh viên ngành khoa học máy tính, chúng em nhận thấy cần trang bị những kiến thức cần thiết về Deep Learning. Vì vậy, mục tiêu của bài tập lớn này của chúng em là tìm hiểu về Deep Learning và cụ thể là tìm hiểu về mạng Google Inception trong bài toán phân loại.
\par Google Inception hay  GoogleNet là mạng biến thể của mạng Convolutional neural network (A deep convolutional neural network architecture \cite{googlenet}). Convolutional neural network (CNN) là một dạng của mạng neuron nhân tạo thông thường (Artificial neural network - ANN) (những mạng này sẽ được chúng em giới thiệu ở phần sau của báo cáo).  
\par Phần tiếp theo của báo cáo bao gồm các phần sau: phần 1 và phần 2 giới thiệu về mạng ANN, CNN; các phần tiếp theo sẽ trình bày về mạng Google Inception và các cải tiến của nó. Cụ thể:
\par \textbf{Phần 1}
\par \textbf{Phần 2} 

\chapter{Giới thiệu mạng neural nhân tạo (ANN)}

\chapter{Giới thiệu mạng CNN}

\chapter{GoogleNet}
Trong phần này chúng em sẽ trình bày về kiến trúc của mạng Google Inception (GoogleNet), nhưng trước hết trong phần đầu tiên chúng em sẽ giới thiệu tổng quan về mạng GoogleNet, mục đích khi thiết kế mạng, những ưu điểm của nó so với các mạng khác trong bài toán phân loại.

\section{Tổng quan về GoogleNet}
Inception là một biến thể của mạng CNN (a deep convolutional neural network architecture \cite{googlenet}), một trong những khác biệt lớn nhất của kiến trúc này là sự cải thiện của tài nguyên tính toán bên trong mạng (mặc dù mạng tăng cả về chiều sâu (số lượng tầng trong mạng) và chiều rộng (số lượng neural trong một tầng) nhưng chi phí tính toán vẫn không đổi (chi phí để học các trọng số trong mạng)) (Mạng GoogleNet trong ILSVRC 2014 (ImageNet Large-Scale Visual Recognition Challenge) có số lượng tham số ít hơn 12 lần với kiến trúc mạng đã chiến thắng năm 2012). 


\section{Inception Module}	



\begin{thebibliography}{9}
\bibitem{googlenet} C. Szegedy, W. Liu, Y. Jia, P. Sermanet, S. Reed, D. Anguelov, D. Erhan, V. Vanhoucke, and A. Rabinovich. \textit{Going deeper with convolutions}

\bibitem{rethinking} C. Szegedy, V. Vanhoucke, S. Ioffe, J. Shlens, and Z. Wojna. \textit{Rethinking the inception architecture for computer vision}

\bibitem{inceptionv4} C. Szegedy, S. Ioffe, and V. Vanhoucke. \textit{Inception-v4, Inception-ResNet and the Impact of Residual Connections on Learning}

\end{thebibliography}

\end{document}
